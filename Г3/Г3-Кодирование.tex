\documentclass[a4paper, 12pt, num=Г3, date = 220.01.2019]{listok}
\usepackage{bm}
\usepackage{scalerel}
\usepackage{tikz}
\usepackage{epigraph}
\usepackage{bussproofs}
\usepackage{multicol}
\usepackage{enumitem}
\usepackage{rotating}
\usepackage{mathtools}
\usetikzlibrary{matrix}
\usetikzlibrary{calc}
\usetikzlibrary{arrows}

\newcommand*{\MP}{\mathrm{MP}}
\renewcommand{\phi}{\varphi}
\DeclareMathOperator{\Dom}{Dom}
%\newcommand*{\hm}[1]{#1\nobreak\discretionary{}{\hbox{$\mathsurround=0pt #1$}}{}} %перед знаком для его дублирования при переносе формулы

\begin{document}
\title{Кодирование}
\maketitle
Наша цель была научиться говорить о выводимости в $\mathrm{PA}$ языком арифметики.
В прошлом листке мы поняли, что рекурсивные функции могут быть хорошим подспорьем в этом деле.
Осталось эту возможность реализовать.

Пусть $\Sigma$  не более чем счётная сигнатура, содержащая функциональные символы $\{f_i^n\}$,
предикатные символы $\{R_i^n\}$, переменные $v_0 , v_1 , \dots$
Например, положим $=$ как $R_0^2$,
$0$ как $f_0^0$, $S$ есть $f_0^1$ и так далее.
Наша цель  приписать гёделевы номера объектам языка, чтобы разным объектам соответствовали разные натуральные числа,
а смысл слова мог бы определяться примитивно-рекурсивным образом.
Обозначив гёделев номер объекта $A$ как $\lceil A \rceil$, распределим номера, скажем, так:
\[
	\lceil v_i   \rceil  \coloneqq \langle 1, i \rangle, 
	\lceil f_i^n \rceil  \coloneqq \langle 2, \langle n, i \rangle \rangle, 
	\lceil R_i^n \rceil  \coloneqq \langle 3, \langle n, i \rangle \rangle, 
	\lceil \neg  \rceil  \coloneqq \langle 4, 0 \rangle, 
	\lceil \to  \rceil  \coloneqq \langle 4, 1 \rangle, 
	\lceil \forallsym  \rceil  \coloneqq \langle 4, 3 \rangle.
\]
\begin{problem}
	Объясняет почему квантору существования и остальным логическим связкам не нужны отдельные гёделевые номера.
\end{problem}
Дальше можно этот язык расширить на более сложные конструкции, например:
$\lceil (A \to B) \rceil = \langle \lceil \to \rceil, \lceil A \rceil, \lceil B \rceil \rangle$,
$\lceil \forall {v_i} A \rceil = \langle \lceil \forallsym \rceil, \lceil v_i \rceil, \lceil A \rceil \rangle$.
\begin{problem}
	Докажите, что $\mathrm{Tm}(x) = \text{<<$x$ есть гёделев номер терма>>}$ является примитивно рекурсивной.
\end{problem}
\begin{problem}
	Докажите, что $\mathrm{AtFm}(x) = \text{<<$x$ есть гёделев номер атомарной формулы>>}$ является примитивно рекурсивной.
\end{problem}
\begin{problem}
	Докажите, что $\mathrm{Fm}(x) = \text{<<$x$ есть гёделев номер формулы>>}$ является примитивно рекурсивной.
\end{problem}
\begin{definition}
	\textit{Нумерал} $\underline{n}$ --- это терм $\underbrace{S(\dots S(0)\dots)}_n$
\end{definition}
\begin{problem}
	Покажите, что $nm(x) \coloneqq \lceil \underline x \rceil$ и $\mathrm{Num}(x) = \text{<<$x$ есть гёделев номер нумерала>>}$ примитивно рекурсивны.
\end{problem}
\begin{problem}
	Докажите, что $\mathrm{Sub}(x, i, y) =$ <<результат подстановки в $x$ выражения $y$ вместо свободных вхождений переменной $v_i$>>
	является примитивно рекурсивной.
	Другими словами, если $x = \lceil \phi \rceil$, то выполняется
	$\mathrm{Sub}(\lceil \phi \rceil, i, \lceil t \rceil) = \lceil \phi [v_i /t]\rceil$.
\end{problem}
\begin{problem}
	Докажите, что $\mathrm{Free}(x, y) =$ <<$x$ есть гёделев номер переменной, имеющей свободное вхождение в выражение с номером $y$>>
	является примитивно рекурсивной.
\end{problem}
\begin{problem}
	Покажите, что следующие предикаты примитивно рекурсивны <<$x$ есть код подформулы формулы с кодом $y$>>,
	<<$t$ подстановочен в $\phi$ вместо свободного вхождения переменной $v_i$>>,
\end{problem}
\begin{definition}
	Пусть $\mathrm{Ax}_i(x) = \text{<<$x$ есть код применения $i$-ой аксиомы $\textrm{Cl}$>>}$,
	$\mathrm{Log}(x) = \bigvee \mathrm{Ax}_i(x)$,
	$\mathrm{MP}(x, y, z) = (y = \langle \lceil \to \rceil, x, z \rangle \& x, y, z \in \mathrm{Fm})$ (выводимость по modus ponens),
	\begin{multline}
		\mathrm{B1}(x, y)
		= (x, y \in \mathrm{Fm}) \&  (\exists{A, a, B, v} (x = \langle A, \lceil \to \rceil, \mathrm{Sub}(B, a, v)) \&\\
		\& (y = \langle A, \lceil \to \rceil, \lceil \forall \rceil, v, B\rangle) \& (A, B \in \mathrm{Fm}) \& v \in \mathrm{Var} \& \mathrm{Tm}(a)) \\
	\end{multline}
	$\mathrm{Gen}(x, i, y) = (y = \langle \lceil \forallsym \rceil, \lceil v_i \rceil, x \rangle)$ (применение квантора всеобщности).
\end{definition}
\begin{problem}
	Покажите, что $\mathrm{Ax}_i$, $\mathrm{Log}$, $\mathrm{MP}$, $\mathrm{Gen}$ являются примитивно рекурсивными.
\end{problem}
\begin{problem}
	Докажите, что $\mathrm{Prf}(x, y) = \text{<<$x$ есть вывод $y$ в языке предикатов>>}$ является примитивно рекурсивной.
\end{problem}
\end{document}
