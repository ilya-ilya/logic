\documentclass[a4paper, 12pt, num=ТА1]{listok}
\usepackage{bm}
\usepackage{scalerel}
\usepackage{bussproofs}
\usepackage{tikz}
\usepackage{epigraph}
\usepackage{bussproofs}
\usepackage{rotating}
\usetikzlibrary{matrix}
\usetikzlibrary{calc}
\usetikzlibrary{arrows}

\newcommand*{\MP}{\mathrm{MP}}
\renewcommand{\phi}{\varphi}
\DeclareMathOperator{\Dom}{Dom}
%\newcommand*{\hm}[1]{#1\nobreak\discretionary{}{\hbox{$\mathsurround=0pt #1$}}{}} %перед знаком для его дублирования при переносе формулы

\begin{document}
\title{Машины Тьюринга}
\maketitle{}
Пусть $\Sigma' = \{0, 1\} \cup \Sigma$ --- конечный алфавит, $\Sigma \ne \emptyset$.
Обозначим через $\Sigma^{\ast}$ множество конечных слов (в том числе пустое) из алфавита $\Sigma$.
Обозначим через $|x|$ длину слова $x$.
Пусть $Q = \{q_0, q_1, \ldots, q_r\}$ --- множество состояний.
\begin{definition}
	Машиной Тьюринга $M$ называется конесное множество команд вида $q a \to q' a' Z$,
	где $q, q' \in Q$, $a, a' \in \Sigma'$, а $Z \in \{-1, 0, 1\}$.
	Такие команды выполняются так: <<если машина находится в состоянии $q$ и видит на ленте символ $a$,
	то она должна перейти в состояние $q'$ поменять текущий символ на $a'$ и сдвинуться в направлении $Z$>>.
	Машина начинает выполнение в \textit{начальном состоянии} $q_1$ и выполнет команды по одной.
	Если машина достигает \textit{конечного (терминального) состояния} $q_0$, то она завершается (halts).
	Программа должна быть \textit{однозначной}: для любых $q \in Q \setminus \{q_0\}$ и $a \in \Sigma'$
	сущесвует ровно одна команда, которая начинается на <<$q a$>>; нет команда, которые начинаются на $q_0 a$.
\end{definition}
\begin{definition}
	Будем говорить, что машина $M$ \textit{применяется к слову} $x \in \Sigma^\ast$,
	если машина $M$ начинает работу на первой букве слова $x \in \Sigma^\ast$ и все остальные ячейки ленты пусты
	(содержат символ 0).
	Если машина $M$ на входе $x$ завершается и на ленте написанно слово $y$, то будем говорить,
	что $y$ --- \textit{результат работы машины} $M$ на входе $x$ (обозначение $M(x) = y$).
	\textit{Временем работы} машины $M$ на входе $x$ называется число шагов, которое $M$ тратит на вычисление результат на слове $x$ (обозначение $T_M(x)$).
\end{definition}
\begin{note}
	Приняты три неформальных соглашения:
	\begin{enumerate}[label=\arabic*.]
		\item При написании программ явно выписывать только <<нетривиальные>> команды отличные от $qa \to qa$.
		\item Натуральные числа представляются на ленте в унарной записи --- число $n$ записывается как $\underbrace{11\ldots1}_{\text{$n$ раз}}$.
		\item В начале работы головка расположена непосредственно слева от входных данных.
			После завершения работы результатом является то слово на ленте (последовательность букв между соседними <<0>>), на которое указывает головка.
	\end{enumerate}
\end{note}
\begin{problem}
	Написать программы для машины Тьюринга:
	\begin{enumerate}
		\item заменяющую во входном слове из 0 и 1 все буквы 0 на 1 и наоборот;
		\item перемещающую 0 через блок единиц ($011\ldots 1 \leadsto 11\ldots 10$);
		\item удвоение блока из $1$;
		\item обращения слова из $0$ и $1$ (пишет на выходе буквы слова в обратном порядке).
		\item сортировка нулей и единиц в двоичном слове
	\end{enumerate}
\end{problem}
\begin{problem}
	Написать программы для машины Тьюринга, вычисляющие следующие функции натурального аргумента:
	\begin{multienum}{2}
		\item $x + 1$;
		\item $x + y$;
		\item $
			f(x) = \left \{
				\begin{aligned}
					x - 1, & x > 0,\\
					0, & x = 0;
				\end{aligned}
				\right .
		$
		\item $|x -y|$;
		\item $2x + 1$;
		\item $2x$;
		\item $[x/2]$;
		\item $xy$.
	\end{multienum}
\end{problem}
\end{document}
