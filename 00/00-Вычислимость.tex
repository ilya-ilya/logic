\documentclass[a4paper, 12pt, num=Г0]{listok}
\usepackage{bm}
\usepackage{scalerel}
\usepackage{tikz}
\usepackage{epigraph}
\usepackage{bussproofs}
\usepackage{multicol}
\usepackage{enumitem}
\usepackage{rotating}
\usetikzlibrary{matrix}
\usetikzlibrary{calc}
\usetikzlibrary{arrows}

\newcommand*{\MP}{\mathrm{MP}}
\renewcommand{\phi}{\varphi}
\DeclareMathOperator{\Dom}{Dom}
%\newcommand*{\hm}[1]{#1\nobreak\discretionary{}{\hbox{$\mathsurround=0pt #1$}}{}} %перед знаком для его дублирования при переносе формулы

\begin{document}
\title{Вычислимость}
\maketitle
\begin{definition}
	$f \colon \N^k \to \N^s$ \textit{вычислима}, если существует машина Тьюринга,
	на любом входе $(n_1 , \dots, n_k)$ вычисляющая $m = f (n_1, \dots, n_k)$.
	Функция $f$ может быть \textit{частичной}, если $\Dom(f) \subsetneq \N^k$; тогда на остальных входах машина Тьюринга зацикливается.
\end{definition}
\begin{definition}
	Множество $P \subseteq \N^k$ \textit{перечислимо},
	если существует вычислимая функция $f$ такая, что область её определения равна $P$.
\end{definition}
\begin{problem}
	Докажите, что $P \subseteq \N^k$ перечислимо тогда и только тогда,
	когда существует вычислимая $f$ такая, что область её значений равна $P$.
\end{problem}
\begin{problem}
	Докажите, что $\emptyset \ne P \subseteq \N^k$ перечислимо тогда и только тогда,
	когда существует тотальная вычислимая $f$ такая, что область её значений равна $P$.
\end{problem}
\begin{problem}[ (Теорема о графике)]
	Докажите, что функция $f$ вычислима тогда и только тогда, когда
	её график $\Gamma_f = \{ (x, y) \mid f(x) = y \}$ перечислим.
\end{problem}
\begin{definition}
	Множество $P \subseteq \N^k$ \textit{разрешимо}, если её характеристическая функция вычислима,
	то есть алгоритм, способный проверить произвольный вход из $\N^k$ на принадлежность $P$.
\end{definition}
\begin{problem}
	Приведите пример перечислимого, но не разрешимого множества.
\end{problem}
\begin{problem}[ (Теорема Поста)]
	Докажите, что $P\subseteq \N^k$ --- разрешимо тогда и только тогда, когда $P$ и $\N^k \backslash P$ одновременно перечислимы.
\end{problem}
\end{document}
