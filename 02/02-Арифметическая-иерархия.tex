\documentclass[a4paper, 12pt, num=Г2]{listok}
\usepackage{bm}
\usepackage{scalerel}
\usepackage{tikz}
\usepackage{epigraph}
\usepackage{bussproofs}
\usepackage{multicol}
\usepackage{enumitem}
\usepackage{rotating}
\usetikzlibrary{matrix}
\usetikzlibrary{calc}
\usetikzlibrary{arrows}

\newcommand*{\MP}{\mathrm{MP}}
\renewcommand{\phi}{\varphi}
\DeclareMathOperator{\Dom}{Dom}
%\newcommand*{\hm}[1]{#1\nobreak\discretionary{}{\hbox{$\mathsurround=0pt #1$}}{}} %перед знаком для его дублирования при переносе формулы

\begin{document}
\title{Арифметическая иерархия}
\maketitle
Будем рассматривать арифметику $\mathrm{PA}(+, \cdot, 0, S, =, \le)$,
где $x \le y \leftrightarrow \exists z x + z = y$.
Введём два обозначения
\begin{align*}
	\forall{ x \le t }\phi(x) & \stackrel{def}{\Leftrightarrow} \forall x (x \le t \to \phi(x)), \\
	\exists{ x \le t }\phi(x) & \stackrel{def}{\Leftrightarrow} \exists x (x \le t \&  \phi(x)),
\end{align*}
где $x$ не входит в $t$.
\begin{definition}
	Класс $\Delta_0$ содержит все \textit{ограниченные формулы},
	то есть такие все вхождения кванторов в которые ограничены.
\end{definition}
\begin{problem}
	Докажите, что если $\phi(\vec x) \in \Delta_0$, то существует алгоритм,
	который по параметрам $\vec n$ проверяет $\mathrm{PA} \models \phi(\vec x)$.
\end{problem}
\begin{definition}
	Определим $\Sigma_n$, $\Pi_n$:
	$\Sigma_0 = \Pi_0 = \Delta_0$,
	$\Sigma_{n + 1} = \{ \exists x \phi(x, y) \mid \phi \in \Pi_n \}$,
	$\Pi_{n + 1} = \{ \forall x \phi(x, y) \mid \phi \in \Sigma_n \}$,
\end{definition}
\begin{definition}
	Будем говорить, что предикат $P \subseteq \N^k$ лежит в $\Sigma_n$ ($\Pi_n$),
	если его можно определить формулой $\Sigma_n$($\Pi_n$)-формулой.
	$P$ принадлежит $\Delta_n$, если он принадлежит и $\Sigma_n$, и $\Pi_n$.
\end{definition}
\begin{problem}
	Докажите, что классы $\Sigma_n$- и $\Pi_n$-предикатов замкнуты относительно ограниченных кванторов.
\end{problem}
\begin{problem}
	Докажите, что $\Sigma_n$ замкнуты относительно квантора существования, а $\Pi_n$ --- всеобщности.
\end{problem}
\begin{problem}
	Докажите, что классы $\Sigma_n$- и $\Pi_n$-предикатов замкнуты относительно объединения, пересечения и дополнения.
\end{problem}
\begin{problem}
	Докажите, что любая арифметическая формула\footnote{формула арифметики $\mathrm{PA}$} эквивалентна некоторой формуле из $\Sigma_n$.
\end{problem}
\begin{theorem}
	Предикат $P \subseteq \N^k$ является $\Sigma_1$-определимым в том и только в том случае, когда $P$ перечислим\footnote{%
	Существует машина Тьюринга, выводящая только элементы $P$ по одному}.
\end{theorem}
\begin{problem}[ (Легкая стрелка теоремы)]
	Докажите, что $\Sigma_1$-предикаты перечислимы.
\end{problem}
Для доказательства обратной стрелки нам понадобится следующий неформальный тезис Чёрча:
\begin{center}
	\textit{Всякая вычислимая функция является частично рекурсивной.}
\end{center}
\begin{problem}
	Докажите, что график любой рекурсивной функции $\Sigma_1$-определим.
\end{problem}
\begin{problem}
	Покажите, что из предыдущей задачи следует обратная стрелка теоремы.
\end{problem}
\end{document}
