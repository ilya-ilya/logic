\documentclass[a4paper, 12pt, num=ЛП]{listok}
\usepackage{bm}
\usepackage{scalerel}
\usepackage{bussproofs}
\usepackage{tikz}
\usepackage{epigraph}
\usepackage{bussproofs}
\usepackage{rotating}
\usetikzlibrary{matrix}
\usetikzlibrary{calc}
\usetikzlibrary{arrows}

\newcommand*{\MP}{\mathrm{MP}}
\renewcommand{\phi}{\varphi}
\DeclareMathOperator{\Dom}{Dom}
%\newcommand*{\hm}[1]{#1\nobreak\discretionary{}{\hbox{$\mathsurround=0pt #1$}}{}} %перед знаком для его дублирования при переносе формулы

\begin{document}
\title{Логика предикатов}
\maketitle{}
Пусть $\Gamma$ --- множество замкнутых формул (\textit{теория}), а $F$ --- замкнутая формула.
\begin{definition}
	Формула $F$ \textit{логически следует} из $\Gamma$ (обозначение $\Gamma \models F$), если для каждой интерпретации,
	обращающей все формулы из $\Gamma$ в истину, формула $F$ также оказывается истинной.
\end{definition}
\begin{definition}
	Формула $F$ \textit{дедуктивно следует или выводима} из множества $Γ$ (обозначение $Γ \vdash F$),
	если у неё существует \textit{формальный вывод} из гипотез $Γ$.
\end{definition}
Напомним, что формальным выводом $F$ из $\Gamma$ называется последовательность утверждений $A_1, A_2, \ldots, A_n = F$,
где каждое $A_i$ либо \textit{аксиома}, либо элемент $\Gamma$, либо получено из предыдущих по \textit{правилам вывода}.
Аксиомы логики предикатов:
\begin{enumerate}[label = \textbf{A\arabic*.}]
\begin{multicols}{2}
	\item $A \to (B \to→ A)$,
	\item $A \wedge B \to A$, $A \wedge B \to B$,
	\item $A \to (B \to A \wedge B)$,
	\item $A \to A \vee B$, $B \to A \vee B$,
	\item $\neg \neg A \to A$,
	\item $(A \to C) \to ((B \to C) \to (A\vee B \to C))$,
	\item $(A \to B) \to ((A \to \neg B) \to \neg A)$,
	\item $(A \to (B \to C)) \to ((A \to B) \to (A \to C))$,
	\item $(\forall{x} A(x)) \to A(t)$, где $t$ --- терм,
	\item $A(t) \to \exists{x} A(x)$, где $t$ ---терм.
\end{multicols}
\end{enumerate}
Правил вывода всего 3: Modus ponens и два правила Бернайса
\begin{multicols}{3}
\begin{prooftree}
	\AxiomC{$A$}
	\AxiomC{$A \to B$}
	\LeftLabel{$\MP$:}
	\BinaryInfC{$B$}
\end{prooftree}
\begin{prooftree}
	\AxiomC{$A \to B(a)$}
	\LeftLabel{B1}
	\UnaryInfC{$A \to \forall{x} B(x)$}
\end{prooftree}
\begin{prooftree}
	\AxiomC{$B(a) \to A$}
	\LeftLabel{B1}
	\UnaryInfC{$(\forall{x} B(x)) \to A$}
\end{prooftree}
\end{multicols}
\begin{theorem}[Полнота и корректность ЛП, Гёдель]
	\[
		T \models F \Leftrightarrow T \vdash F.
	\]
\end{theorem}
\begin{theorem}[Гёделя о компактности]
	Теория $T$ непротиворечива\footnote{Теория непротиворечива, когда у неё есть модель, то есть интерпретация сигнатуры, в которой эта теория верна}
	тогда и только тогда, когда любая её конечная подтеория непротиворечива.
\end{theorem}
\end{document}
