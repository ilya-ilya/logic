\documentclass[a4paper, 12pt, num=Г5]{listok}
\usepackage{bm}
\usepackage{scalerel}
\usepackage{tikz}
\usepackage{epigraph}
\usepackage{bussproofs}
\usepackage{multicol}
\usepackage{enumitem}
\usepackage{rotating}
\usepackage{mathtools}
\usetikzlibrary{matrix}
\usetikzlibrary{calc}
\usetikzlibrary{arrows}

\newcommand*{\MP}{\mathrm{MP}}
\renewcommand{\phi}{\varphi}
\DeclareMathOperator{\Dom}{Dom}
%\newcommand*{\hm}[1]{#1\nobreak\discretionary{}{\hbox{$\mathsurround=0pt #1$}}{}} %перед знаком для его дублирования при переносе формулы

\begin{document}
\title{Теоремы Гёделя}
\maketitle
\begin{definition}
	Введём предикат \textit{доказуемости} в арифметике
	\[
		\mathrm{Pr}(x) = \exists y \mathrm{Prf}(y, x).
	\]
\end{definition}
\begin{theorem}[Первая теорема Гёделя о неполноте]
	Существует замкнутая формула $\phi$ такая, что $\N \models \phi$, $\mathrm{PA} \not \vdash \phi$.
\end{theorem}
\begin{problem}[(Лемма о неподвижной точке)]
	Пусть $\psi(x_i)$ --- формула с единственной свободной переменной $x_i$.
	Докажите\footnote{Ответ: $\theta(x) = \psi(\mathrm{Sub}(x,i,x))$, $m = \lceil \theta \rceil$, $\phi = \theta(\underline m)$},
	что найдётся такая арифметическая замкнутая формула $\phi$, что
	\[
		\mathrm{PA} \vdash \phi \leftrightarrow \psi(\lceil \phi \rceil).
	\]
\end{problem}
\begin{problem}
	Применив лемму о неподвижной точке к $\neg \mathrm{Pr}$ докажите первую теорему Гёделя.
\end{problem}
\begin{definition}
	Рассмотрим \textit{формулу непротиворечивости арифметики Пеанно}
	\[
		\mathrm{Consis} = \neg \mathrm{Pr}(\lceil 0 = 1 \rceil)
	\]
\end{definition}
Введём обозначения: $\perp = (0 = 1)$, $\Box \psi = \mathrm{Pr}(\lceil \psi \rceil)$.
Следующие теоремы доказываются одна за другой.
\begin{theorem}[Гильберт, Бернайс, Гёдель, Леб]
	Для любых замкнутых $\phi$, $\psi$ верно, что
	\begin{align*}
		\mathrm{PA} \vdash \phi &\Rightarrow \mathrm{PA} \vdash \Box \phi \\
		\mathrm{PA} &\vdash \Box(\phi \to \psi) \to (\Box \phi \to \Box \psi) \\
		\mathrm{PA} &\vdash \Box \phi \to \Box \Box \phi
	\end{align*}
\end{theorem}
\begin{theorem}[Вторая теорема Гёделя о неоплноте]
	\[
		\mathrm{PA} \not \vdash \mathrm{Consis}.
	\]
\end{theorem}
\begin{theorem}[Леб]
	Если $\mathrm{PA} \vdash \Box \phi \to \phi$, то $\mathrm {PA} \vdash \phi$.
\end{theorem}
\begin{theorem}[Тарского]
	Не существует формулы $T(x)$ такой, что для всех замкнутых формул $\phi$
	\[
		\N \models \phi \Leftrightarrow \N \models T(\lceil \phi \rceil).
	\]
	Эквивалентная формулировка: одноместный предикат <<$x$ есть гёделев номер формулы, истинной в $\N$>> не является арифметическим.
\end{theorem}
\begin{theorem}
	$Q$ и $\mathrm{PA}$ неразрешимы.
\end{theorem}
\begin{theorem}
	Исчисление предикатов в сигнатуре арифметики неразрешимо.
\end{theorem}
\end{document}
