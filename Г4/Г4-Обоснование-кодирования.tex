\documentclass[a4paper, 12pt, num=Г4]{listok}
\usepackage{bm}
\usepackage{scalerel}
\usepackage{tikz}
\usepackage{epigraph}
\usepackage{bussproofs}
\usepackage{multicol}
\usepackage{enumitem}
\usepackage{rotating}
\usepackage{mathtools}
\usetikzlibrary{matrix}
\usetikzlibrary{calc}
\usetikzlibrary{arrows}

\newcommand*{\MP}{\mathrm{MP}}
\renewcommand{\phi}{\varphi}
\DeclareMathOperator{\Dom}{Dom}
%\newcommand*{\hm}[1]{#1\nobreak\discretionary{}{\hbox{$\mathsurround=0pt #1$}}{}} %перед знаком для его дублирования при переносе формулы

\begin{document}
\title{Обоснование кодирования}
\maketitle
Пусть $T$ некоторая теория в языке арифметики.
\begin{definition}
	Отношение $P(\vec{x}) \subseteq \N^k$ \textit{разрешимо} в $T$,
	если существует формула $\phi(\vec x)$ , что $\forall{\vec n}$
	\begin{align*}
		P(\vec n) & \Rightarrow T \vdash \phi(\underline{\vec n}) \\
		\neg P(\vec n) & \Rightarrow T \vdash \neg \phi(\underline{\vec n })
	\end{align*}
\end{definition}
\begin{definition}
	Функция $f(\vec x )$ \textit{представима} в $T$, если существует $\phi (\vec x , y)$ такая, что:
	\begin{gather*}
		f(\vec n) = m \Rightarrow T \vdash \phi(\underline{\vec n}, \underline{m}),\\
		T \vdash \forall y (\phi(\underline{\vec n}, y) \to y = \underline{m}).
	\end{gather*}
\end{definition}
Для того, чтобы можно было формулировать в арифметике утверждения, подобные теоремам Гёделя (о доказуемости чего-либо в арифметике)
нам понадобиться следующая теорема, обосновывающая тот синтаксис, который мы ввели в предыдущем листке.
\begin{theorem}[Обоснования]
	\begin{itemize}
		\item Любая ПРФ $\Sigma_1$-определима в $\N$;
		\item Рассмотрим $\mathrm{PA}$, в которой схему индукции заменим аксиомой\footnote{Такая
			теория называется \textit{арифметикой Робинсона} $Q$.
			В ней, в отличии от $\mathrm{PA}$ конечное число аксиом.} $\forall y (y \ne 0 \to \exists x y = S(x)$.
			В $Q$ разрешимы все ПР отношения и представимы все ПРФ.
	\end{itemize}
\end{theorem}
\begin{definition}
	Функция $\beta(x, y, z) = rm((z + 1) \cdot y + 1, x)$ называется $\beta$ функцией Гёделя.
\end{definition}
\begin{problem}
	Покажите, что $\beta$ является $\Sigma_1$-определимой.
\end{problem}
\begin{problem}
	Пусть $m = \max(n, k_0, \dots, k_n)$, $c = m!$, $u_i = c(i + 1) + 1$.
	Покажите, что $(u_i, u_j) = 1$, если $i \ne j$.
\end{problem}
\begin{theorem}[Китайская теорема об остатках]
\[
	\forall{\{k_0, \dots, k_n\}} \exists{b < u_1, \dots, u_n} \forall{i \le n} b \equiv k_i (\mathrm{mod}\; u_i)
\]
\end{theorem}
\begin{problem}
	Докажите, что
	\[
		\forall{\{k_0, \dots, k_n\}} \exists{a, b} \forall{i \le n} \beta(a, b, i) = k_i.
	\]
\end{problem}
\begin{problem}
	Докажите, что любая ПРФ $\Sigma_1$-определима в $\N$.
\end{problem}
\begin{problem}
	Докажите, что всякая $\Delta_0$-формула разрешима в $Q$.
\end{problem}
\begin{problem}
	Докажите, что всякая вычислимая формула представима в $Q$.
\end{problem}
\begin{problem}
	Докажите, что всякая $\Sigma_1$-формула разрешима в $Q$.
\end{problem}
\begin{problem}
	Завершите доказательство второго пункта теоремы Обоснования.
\end{problem}
\end{document}
